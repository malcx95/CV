\documentclass[a4paper,notitlepage]{article}
\usepackage[utf8]{inputenc} %Make sure all UTF8 characters work in the document
\usepackage{listings} %Add code sections
\usepackage{color}
\usepackage[yyyymmdd]{datetime}
\usepackage{graphicx}
\usepackage{titling}
\usepackage{titlesec}
\usepackage{listliketab}
\usepackage{textcomp}
\usepackage[hyphens]{url}
\usepackage[bottom]{footmisc}
\definecolor{listinggray}{gray}{0.9}
\definecolor{lbcolor}{rgb}{0.9,0.9,0.9}
\usepackage{geometry}
\geometry{margin=3cm}
\usepackage{parskip} 

\hyphenation{regres-sions-test-er}

\renewcommand{\dateseparator}{--}
\renewcommand{\arraystretch}{1.3}
\titlespacing*\section{0pt}{10pt plus 4pt minus 2pt}{0pt plus 2pt minus 2pt}
\titlespacing*\subsection{0pt}{10pt plus 4pt minus 2pt}{0pt plus 2pt minus 2pt}
\pretitle{%
\begin{center}
	\includegraphics[width=3cm]{bild.jpeg}\\[\bigskipamount]
	% TODO hitta en bättre bild
}
\posttitle{\end{center}}
\title{
\huge{CV - Malcolm Vigren}\vspace{-3ex}}
\date{\today}
\begin{document}
	\maketitle
\underline{Malcolm} John Shubi Vigren, 19950127-0970

Batterigatan 9

587 50 Linköping

E-post: \underline{trekommafem2 \textbf{at} gmail.com}

Mobil: 072-534 16 81

\section*{Arbetslivserfarenhet}
\noindent\begin{tabular}{@{}l p{13cm}}

\textbf{2018} & Labbassitent och seminarieledare på hösten i kursen TDDE23/24 på Linköpings
    Universitet. \\

\textbf{2018} & Extrajobb och sommarjobb på Veoneer i Linköping, från april till augusti.
    Arbetade i teamet ansvarig för datahantering, "Team Se7en", på RA Simulation. \\

\textbf{2017} & Labbassistent och seminarieledare på hösten i kursen TDDE23/24 på Linköpings
    Universitet. \\

\textbf{2017} & Sommarjobb på Autoliv i Linköping. Utvecklade verktyg 
    för behandling av data och lagra dessa i en databas i C\# och Microsoft SQL Server. \\

\textbf{2016} & Sommarjobb på Autoliv i Linköping. Utvecklade verktyg för
    utförande av regressionstester i Python och C\#.\\

\textbf{2015 - 2016:} & Butiksarbetare på ICA Supermarket Eneby i Norrköping, med
arbetsuppgifter som kassör, spelombudsbiträde och postombudsbiträde, vid sidan
av universitetsstudier. \\

\textbf{2013:} & Skapade en animerad film åt SundaHus för en mässa.
\\

\textbf{2012 - 2015:} & Butiksarbetare på Hemköp Ljungsbrohallen,
med arbetsuppgifter
som kassör,
spelombudsbiträde, postombudsbiträde, ansvarig för ändring av priser med mera.
\\

\textbf{2010:} & PRAO i fem dagar på Hemköp Ljungsbrohallen under
höstterminen. \\

\textbf{2010:} & PRAO i tre dagar på ABB Service-verkstaden i
Norrköping på vårterminen. \\

\textbf{2010 - 2013:} & Städning av judolokalen i Linköping en gång
i veckan. \\

\textbf{2008 - 2012:} & Inmatning av produktinformation i SundaHus Miljödatabas under
skolloven.	\\

	\end{tabular}

\section*{Utbildning}
\noindent\begin{tabular}{@{}l p{11cm}}
	\textbf{Högskola:} & \textit{Linköpings Universitet} - Civilingenjör i
	datateknik 300hp - Pågående, påbörjad HT2014 \\

	\textbf{Gymnasieskola:} & \textit{Berzeliusskolan} - Teknikvetenskap -
	Avslutad, examen 2014 \\

	\textbf{Grundskola:} & \textit{Internationella Engelska Skolan i Linköping},
	årskurs 6-9 \\

	\textbf{Grundskola:} & \textit{Brunnbyskolan}, årskurs 1-5 \\
	\end{tabular}

%\subsection*{Avslutade universitetskurser}
%\noindent\begin{tabular}{@{}l l l l}
%	\textbf{TDDD84} & \textit{Ingenjörsprofessionalism, del 3} & 1 hp & Betyg 5 \\
%	\textbf{TATA24} & \textit{Linjär algebra} & 8 hp & Betyg 5 \\
%	\textbf{TSEA82} & \textit{Datorteknik} & 4 hp & Betyg G \\
%	\textbf{TDDD78} & \textit{Objektorienterad programmering och Java} & 6 hp & Betyg 5 \\
%	\textbf{TDDD94} & \textit{Ingenjörsprofessionalism, del 4} & 1 hp & Betyg 5 \\
%	\textbf{TGTU50} & \textit{Industrikunskap} & 1.5 hp & Betyg D (deltagit) \\
%	\textbf{TSEA22} & \textit{Digitalteknik} & 6 hp & Betyg 4 \\
%	\textbf{TSTE21} & \textit{Elektronik} & 5 hp & Betyg G \\
%	\textbf{TATA42} & \textit{Envariabelanalys 2} & 6 hp & Betyg 3 \\
%	\textbf{TATA41} & \textit{Envariabelanalys 1} & 6 hp & Betyg 4 \\
%	\textbf{TATA79} & \textit{Inledande matematisk analys} & 6 hp & Betyg 3 \\
%	\textbf{TDDD70} & \textit{Ingenjörsprofessionalism del 1} & 1 hp & Betyg 5 \\ 
%	\textbf{TDDD63} & \textit{Perspektiv på datavetenskap} & 7 hp & Betyg G \\
%	\textbf{TDDD73} & \textit{Funktionell och imperativ programmering i Python} & 4 hp &
%	Betyg 4 \\
%	\textbf{TATA65} & \textit{Diskret Matematik} & 4 hp & Betyg 4 \\
%	\textbf{TDDC66} & \textit{Datorsystem och programmering} & 4 hp & Betyg G \\
%	\end{tabular}
%\subsection*{Kurser avslutade till VT2016}
%\noindent\begin{tabular}{@{}l l l}
%	\textbf{TDDD86} & \textit{Datastrukturer, algoritmer och
%programmeringsparadigm} & 11 hp \\
%	\textbf{TFYY68} & \textit{Mekanik} & 6 hp \\
%	\textbf{TATA76} & \textit{Flervariabelanalys} & 4 hp \\
%	\textbf{TSEA83} & \textit{Datorkonstruktion} & 8 hp \\
%	\textbf{TDDB68} & \textit{Processprogrammering och operativsystem} & 6 hp \\
%	\textbf{TDDD98} & \textit{Ingenjörsprofessionalism del 6} & 1 hp \\
%	\textbf{TAMS27} & \textit{Matematisk statistik} & 6 hp \\
%	\textbf{TSRT04} & \textit{Introduktionskurs i Matlab} & 2 hp \\
%	\textbf{TFYA86} & \textit{Fysik} & 6 hp \\
%	\end{tabular}

\section*{Programmeringskunskaper}
Goda kunskaper inom C, C\#, C++, Java, Python, VHDL, Matlab och assembler.
Grundläggande kunskaper inom HTML, JavaScript, Haskell, BASH-skript,
CoffeeScript och Ruby On Rails. Goda kunskaper inom versionshanteringsverktyget
Git, texteditorn VIM, LaTeX och Deep Learning-biblioteket PyTorch.

Goda kunskaper inom bildbehandling, datorseende och maskininlärning.

\section*{Språkkunskaper}
Flytande svenska och engelska i tal och skrift. Gymnasiekunskaper inom tyska
(Tyska steg 3).

\section*{Profil och intressen}
Jag är systematisk, noggrann och lättlärd. Jag är engagerad i mitt arbete och
försöker alltid att göra mitt bästa. Mina huvudintressen är programmering, vetenskap, 
teknik, matematik, datorer, motorer och bilar. Jag spenderar en stor del av min fritid på
hobbyprogrammering.

\section*{Övriga meriter}
Körkort (AM- samt B-behörighet).

Var med i vinnarlaget i en av tävlingarna under LiTHe Kods Jubileumshack den
12-13:e februari 2016, i vilken en algoritm för möblering och visualisering av 
ett rum skulle utvecklas över natten.

Erhöll 1000 kr ur Berzeliusskolans stipendiefond vid examen för mina höga
betyg.

Konstruerade och programmerade ett mekaniskt RGB-bakbelyst tangentbord som
hobbyprojekt under hösten 2016. Detta, och många andra av mina hobby- och skolprojekt, finns på min Github-sida: \url{https://github.com/malcx95}.

Var projektledare i projektarbetet under kursen \textit{Konstruktion med
mikrodatorer} (kurskod TSEA29) under höstterminen 2016.
I projektet skulle 7 projektmedlemmar
konstruera en sexbent robot, med en tidsbudget på 160 timmar per person. Koden
och dokumentationen för projektet finns på Github:
\url{https://github.com/TheZoq2/LiTHe-Hex}.

\end{document}
